\documentclass{article}
\usepackage{graphicx} % Required for inserting images
%Hyphenation rules
\usepackage{amsfonts}
\usepackage{amsmath}
\usepackage[left=2cm,right=2cm,top=1cm,bottom=2cm]{geometry}

%\vspace{1cm}
\usepackage[english, russian]{babel}

\title{Матанчик}
\author{Георгий Ежелев P3132}

\begin{document}

\maketitle

\section{№12}
Введем функцию \textbf{ k(x) = f(x) -g(x)}\\
Тогда k - непрерывна на [a, b] и k(a) < 0, а k(b) > 0. Следовательно по 1 теореме Больцано-Коши  \(\exists c \in (a, b): k(c) = 0 \)=> f(x) = g(x), x\(\in[a,b]\), т.к. по условию в пограничных точках неравенство строгое.\\ 
\section{№11}
Пусть это не так и \(\exists f(x)\) непрерывная и инъективная, но не монотонная.\\
Тогда возможны 2 случая:
\subsection{ }
\(\exists a,b,c: a<b,f(a)<f(b);a<c,f(a)>f(c)\)Тогда, следуя из свойств непрерывности получаем, что \([f(a),f(b)] \subset f([a,b])\), и \([f(a),f(с)] \subset f([a,с])\), следовательно на отрезке [b, c] функция не удовлетворяет условию инъективности, следовательно получаем противоречие => \textbf{функция монотонна}
\subsection{ }
Докажем что она строго монотонна.\\
Аналогично, пусть \(\exists a,b: f(a)=f(b)\). Функция сразу противоречит условию инъективности, значит \textbf{она строго монотонна}.\\
\subsection{№14}
Если функция непрерывна, то для \(\forall x,x_0 \in X: \exists U(x_0): x\in U(x_0) f(x) \in V(f(x))\)Следовательно у каждого значения функции есть окрестности, полностью принаждлежащие так же области значений функции,следовательно между любыми двумя значеняими функции всегда найдутся еще из их окрестностей, следовательно разрывов не будет => значит отрезок один.\\
В каждой окрестности \(x_0 \in X\) функция ограничена(иначе она не непрерывна). Тогда из всех этих окрестностей можно выделить конечное покрытие, тогда можно предъявить точную верхнюю грань для ограниченного множества значений функции. НЕДОСТАТОЧНО
\subsection{№13}
Докажем в обе стороны:\\
\subsubsection{}
Если f(x) = cx, то f(x+y) = c(x+y) = cx + cy = f(x) + f(y); cx и cy непрерывны (т.к. с - константа, x,y - непрерывны) => f(x) - непрерывно.
\subsubsection{}
Наоборот, если f(x) непрерывна и \(f: \mathbb{R} -> \mathbb{R}\) f(x + y) = f(x) + f(y) 3 случая либо Ker(f) = {0} либо ядро включает другой набор элементов, либо ядро включает \(\forall x \in X\).\\
1)Ker(f) = {0}. Тогда это инъекция(ЗДЕСЬ РАЗВЕРНУТЬ, тогда из задачи 11 => инъекция+непрерывность = строгая монотонность.\\
Пусть y = f(x) <=> f(y) = f(f(x)) <=> f(y) - f(f(x)) = 0 <=> f(y - f(x)) = 0\\
Т.к. f(0) = 0 (по свойству мономорфизма) то f(f(x)) = f(0 + f(x)) = f(0) + f(x) = 0 + f(x) = f(x), тогда f(y) - f(f(x)) = f(y) - f(x) = f(y - x) = 0 <=> y = x. чтд\\
2)Ker(f) = X. Тогда эта функция f(x) = 0x, c = 0.\\
3)Ker(f) != {0}

f(xy) = f(x+x+x...(y times)) = f(x) + f(x) + f(x)...y раз = f(y)   

\section{15}
Раз функция возрастающая, то \(f(x_n) > x_n\) следовательно последовательность x_n возрастающая и ограничена на [a,b] значит существует супремум на [a,b]. Т.к. функция не может быть больше супремума (иначе это не супремум), то \(f(sup(x_n))\) <= sup(x_n), т.к. f - возрастает то \(f(sup(x_n)) = sup(x_n)\)неподвижная точка и предел (в силу свойств супремума).
\end{document}

